\newpage
\section{Инструкция}

\begin{center}
	о заполнении журнала по производственной практике
\end{center}

Журнал по производственной практике студентов имеет единую форму для всех видов практик.

Задание в журнал вписывается руководителем практики от института в первые три – пять дней пребывания студентов на практике в соответствии с тематикой, утвержденной на кафедре до начала практики. Журнал по производственной практике является основным документом для текущего и итогового контроля выполнения задания, требований инструкции и программы практики.

Табель прохождения практики, задание, а также технический отчет выполняются каждым студентом самостоятельно.
Журнал заполняется студентом непрерывно в процессе прохождения всей практики и регулярно представляется для просмотра руководителем практики. Все их замечания подлежат немедленному выполнению.

В  разделе «Табель прохождения практики» ежедневно должно быть указано, на каких рабочих местах и в качестве кого фактически работал студент. Эти записи проверяются и заверяются цеховыми руководителями практики, в том числе мастерами и бригадами. График прохождения практики заполняется в соответствии с графиком распределения студентов по рабочим местам практики, утвержденным руководителем предприятия.

В разделе «Рационализаторские предложения» должно быть проведено содержание поданных в цехе рационализаторских предложений со всеми необходимыми расчетами и эскизами. Рационализаторские предложения подаются индивидуально и коллективно.

Выполнение студентами задания по общественно-политической практике заносится в раздел «Общественно-политическая практика». Выполнение работы по оказанию практической помощи предприятию (участие в выполнении спец. заданий, работа сверхурочно и т.п.) заносится в раздел журнала «Работа в помощь предприятию» с последующим письменным подтверждением записанной работы соответствующими цеховыми руководителями.

Раздел «Технический отчет по практике» должен быть заполнен особо тщательно. Записи необходимо делать чернилами в сжатой, но вместе с тем четкой, и ясной форме и технически грамотно. Студент обязан ежедневно подробно излагать содержание работы, выполняемой за каждый день. Содержание этого раздела должно отвечать тем конкретным требованиям, которые предъявляются к техническому отчету заданием и программой практики. Технический отчет должен показать умение студента критически оценивать работу данного производственного участка и отразить, в какой степени студент способен применить теоритические знания для решения конкретных производственных задач.

Иллюстративный и другие материалы, использованные студентом в других разделах журнала, в техническом отделе не должны повторяться, а следует  ограничиваться лишь ссылкой на него. Участие студентов в производственно-технической конференции, выступления с докладами, рационализаторские предложения и т.п. должны заноситься на свободные страницы журнала.


Примечание. Синьки, кальки и другие дополнения к журналу могут быть сделаны только с разрешения администрации предприятия и должны подшиваться в конце журнала, но лучше обходиться без них.

Руководители практики от института обязаны следить за тем, чтобы каждый цеховой руководитель практики перед уходом студентов из данного цеха в другой цех вписывал в журнал студента  отзывы об их работе в цехе.

Текущий контроль работы студента осуществляется руководителем практики от института и цеховыми руководителями практики заводов. Все замечания студентам руководители делают в письменном виде на страницах журнала, ставя при этом свою подпись и дату проверки. Результаты защиты технического отчета заносятся в протокол и одновременно заносятся в ведомость и зачетную книжку студента.

Примечание. Нумерация чистых страниц журнала проставляется каждым студентом в своем журнале до начала практики.
 

С инструкцией о заполнении журнала ознакомился:

\vspace{1em}

«$\quad$» $\underline{\qquad \qquad}$ 20$\qquad$г.	$\qquad \qquad$			Студент $\underline{\qquad \qquad}$